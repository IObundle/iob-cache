% SPDX-FileCopyrightText: 2024 IObundle
%
% SPDX-License-Identifier: MIT

The above paragraph describes a desirable simulation setup, but IOb-Cache's
simulation environment still lacks a modular simulation structure. Currently,
only a set of primary non-pipelined write followed by read tests is
implemented. However, IOb-Cache has been thoroughly verified in-system, with two
cache levels. Various open-source RISC-V processors have proven that IOb-Cache
works well: PicoRV32, SSRV, VexRISCV, and DarkRV.

The testbench architecture involves the following components and data flow:

\begin{itemize}
  \item Includes an iob\_universal\_converter module to test the various types of interfaces supported by the IOb-Cache front-end interface.
  \item The IOb-Cache core's back-end interface is connected to a model memory via AXI bus.
\end{itemize}

The testbench controller orchestrates the test sequence as follows:

\begin{enumerate}
  \item Initializes cache.
  \item Simple test: Writes a few random words to cache and reads them back for verification. % simple test
  \item Data test: Writes custom data words to cache and reads them back for verification. % data test
  \item Address test: Writes words to specific addresses, including the highest one, and reads them back for verification. % address test
  \item LRU test: Writes words to specific addresses at regular interval and reads them back to verify cache replacement policy. % LRU test
  \item Controller test: Reads and modifies Cache controller CSRs to verify their functionality. % Controller test
\end{enumerate}
